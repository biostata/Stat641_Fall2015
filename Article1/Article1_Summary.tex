\documentclass[11pt,a4paper]{article}

%%%%%%%%%%%%%%%%%%%%%%%%%%%%%%%%%%%%%%%%%%%%%%%%%%%%%%%%%%%%%%%%%%%%%%
%% Input header file 
%%%%%%%%%%%%%%%%%%%%%%%%%%%%%%%%%%%%%%%%%%%%%%%%%%%%%%%%%%%%%%%%%%%%%%
\input{HeaderfileTexDocs}

%%%%%%%%%%%%%%%%%%% To change the margins and stuff %%%%%%%%%%%%%%%%%%%
\geometry{left=0.9in, right=0.9in, top=1in, bottom=0.8in}
%\setlength{\voffset}{0.5in}
%\setlength{\hoffset}{-0.4in}
%\setlength{\textwidth}{7.6in}
%\setlength{\textheight}{10in}
%%%%%%%%%%%%%%%%%%%%%%%%%%%%%%%%%%%%%%%%%%%%%%%%%%%%%%%%%%%%%%%%%%%%%%%
\begin{document}

\title{Article Summary 1}
\author{Subhrangshu Nandi\\
  Stat 641; Fall 2015}
\date{October 1, 2015}
%\date{}

\maketitle
% --------------------------------------------------------------
%                         Start here
% --------------------------------------------------------------

\noindent
\textbf{Background}

While it has been demonstrated that lowering LDL cholesterol levels helps in the prevention of major cardiovascular events and stroke, it is not as clear whether reducing LDL levels much lower than 100mg per deciliter in patients with CHD provides excess benefits. As death due to cardiovascular disease is a major public health issue, any effort to reduce the rate of events due to cardiovascular disease is of great interest. This study uses a statin drug as a proxy for lower LDL levels. The implicit assumption of the study is that since atorvastatin is known to reduce LDL cholesterol, if patients are randomized according to atorvastatin dosage level, since the higher dose will lower LDL levels more and hence outcomes are assumed to be caused by the lower levels of LDL. This assumption is very strong and hence all results should be viewed skeptically. In reality, patients are randomized to dosage level and hence any conclusions made can only be related to the dosage level and not LDL levels. \\

\noindent
\textbf{Study Design}

Eligable participants to be included in the study are men and women from 35 to 75 years of age with objectively determined CHD, defined as at least one of: previous myocardial infarction, previous or current angina with objective evicdence of atherosclerotic CHD, and a history of coronary revascularization. 

The following conditions are the exclusion criteria: 
\begin{itemize}
\item hypersensitivity to statins, 
\item active liver disease or hepatic dysfunction defined as alanine aminotransferase or aspartate aminotransferase $>$ 1.5 times the upper limit of normal, 
\item women who are pregnant or breast-feeding, 
\item patients with nephrotic syndrome, 
\item patients with uncontrolled diabetes mellitus, 
\item patients with uncontrolled hypothyroidism, 
\item patients with uncontrolled hypertension (as defined by the investigator) at the screening visit, 
\item patients with a myocardial infarction, coronary revascularization procedure or severe/unstable angina within 1 month of screening, 
\item patients with any planned surgical procedure for the treatment of atherosclerosis, an ejection fraction $< 30$ percent, 
\item patients with hemodynamically important valvular disease, 
\item patients with gastrointestinal disease limiting drug absorption or partial ileal bypass, 
\item patients with any nonskin malignancy, unexplained creatine phosphokinase levels $> 6$ times the upper limit of normal, 
\item patients with concurrent therapy with long-term immunosuppresants, 
\item patients with concurrent therapy with lipid-regulating drugs not specified as study treatment in the protocol, 
\item patients with history of alcohol abuse, and 
\item patients with participation in another clinical trial concurrently or within 30 days before screening. 
\end{itemize}

18,469 patients were screened, 8,466 were excluded. The remaining 10,003 underwent randomization. Patients were randomly assigned, in a double-blind fashion, to either 10mg or 80mg of atorvastatin per day.  Patients were then followed up at week 12, months 6, 9, and 12 and every 6 months thereafter. 

The study had a power of 85 percent to detect an absolute reduction of 17 percent in the five-year cumulative rate of the primary efficacy outcome in the 80mg group, as compared with the 10mg group with a type I error rate of no more than 0.05. The study had Power of 40 percent to detect a 10 percent reduction in the risk of death from any cause type I error rate of no more than 0.05. 

The primary outcomes measured were occurrences of  major cardiovascular events (i.e. death from CHD, nonfatal non-procedure related myocardial infarction,  resuscitation after cardiac arrest, fatal or nonfatal stroke. The secondary outcomes measured were occurrences of a cerebrovascular event, hospitalization for congestive heart failure, peripheral-artery disease, death from any cause, any cardiovascular event, and any coronary event.\\

\noindent
\textbf{Efficacy Results}

Using a Cox proportional hazards model, the hazard ratio for all primary outcomes is shown to be significantly different from zero, specifically lower than zero [0.78 (0.69-0.89)], indicating that the assignment of 80mg of atorvastatin caused a lower rate of primary events. Similarly, the hazard ratio for all secondary outcomes is shown to be significantly different from zero, specifically lower than zero [0.80 (0.69-0.92)], indicating that the assignment of 80mg of atorvastatin caused a lower rate of primary events. Unfortunately, there was shown to be no significant different between the two groups in terms of mortality, which is arguably the most important outcome. Perhaps most importantly, this study was funded by Pfizer, former owner of the patent of atorvastatin.
 
\end{document}
