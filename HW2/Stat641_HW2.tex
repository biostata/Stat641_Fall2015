\documentclass[11pt,a4paper]{article}

%%%%%%%%%%%%%%%%%%%%%%%%%%%%%%%%%%%%%%%%%%%%%%%%%%%%%%%%%%%%%%%%%%%%%%
%% Input header file 
%%%%%%%%%%%%%%%%%%%%%%%%%%%%%%%%%%%%%%%%%%%%%%%%%%%%%%%%%%%%%%%%%%%%%%
../../../TexScripts/HeaderfileTexDocs.tex

%%%%%%%%%%%%%%%%%%% To change the margins and stuff %%%%%%%%%%%%%%%%%%%
\geometry{left=0.9in, right=0.9in, top=1in, bottom=0.8in}
%\setlength{\voffset}{0.5in}
%\setlength{\hoffset}{-0.4in}
%\setlength{\textwidth}{7.6in}
%\setlength{\textheight}{10in}
%%%%%%%%%%%%%%%%%%%%%%%%%%%%%%%%%%%%%%%%%%%%%%%%%%%%%%%%%%%%%%%%%%%%%%%
\begin{document}

\title{Homework 2}
\author{Subhrangshu Nandi\\
  Stat 641; Fall 2015}
\date{October 1, 2015}
%\date{}

\maketitle

\noindent
\textbf{Problem 1} 
\begin{itemize}
\item[(a)] Let $\beta$ be the true treatment effect. Ignoring $W$, following is a test of the hypothesis $H_0: \beta = 0$
\begin{table}[H]
\centering
\begin{tabular}{rrrrr}
  \hline
 & Estimate & Std. Error & t value & p-value \\ 
  \hline
Intercept & 6.7826 & 0.2960 & 22.91 & $\sim 0$ \\ 
z1 & \textbf{0.5984} & 0.4186 & 1.43 & \textbf{0.1561} \\ 
   \hline
\end{tabular}
\caption{Summary of \texttt{$x \sim z$}}
\end{table}
With a p-value of $0.1561$, we conclude that there is no enough evidence to reject the null hypothesis.

\item[(b)]
Adjusting by \emph{W} the results are:
\begin{table}[H]
\centering
\begin{tabular}{rrrrr}
  \hline
 & Estimate & Std. Error & t value & p-value \\ 
  \hline
Intercept & 5.4793 & 0.2372 & 23.10 & $\sim 0$ \\ 
  z1 & \textbf{0.7831} & 0.2109 & 3.71 & \textbf{0.0003} \\ 
  w2 & -0.3908 & 0.2983 & -1.31 & 0.1933 \\ 
  w3 & 0.7784 & 0.3023 & 2.57 & 0.0116 \\ 
  w4 & 4.0343 & 0.2839 & 14.21 & $\sim 0$ \\ 
   \hline
\end{tabular}
\caption{Summary of \texttt{$x \sim z + w$}}
\end{table}

With this, $\hat{\beta}=0.7831$ and the t-test for $H_0:\beta=0$ is $3.71$ with an associated p-value close
to zero. We reject the null hypothesis and conclude that there is a treatment difference, once adjusted for the levels of $W$.

\item[(c)]
From (b), we have the following linear model: 
$$X_i=\alpha+\beta z_i+\gamma_2 I\{w_i=2\}+\gamma_3 I\{w_i=3\}+\gamma_4 I\{w_i=4\}+\epsilon_i,$$
where $\epsilon_i\stackrel{iid}{\sim}N(0,\sigma^2)$ and $i=1,\ldots,n$.
Hence,
\begin{eqnarray*}
\text{If }z_i=0&:& \mu_0=E(X_i|w_i)=\alpha+\gamma_2 I\{w_i=2\}+\gamma_3 I\{w_i=3\}+\gamma_4 I\{w_i=4\}\\
\text{If }z_i=1&:& \mu_1=E(X_i|w_i)=\alpha+\gamma_2 I\{w_i=2\}+\gamma_3 I\{w_i=3\}+\gamma_4 I\{w_i=4\}+\beta
\end{eqnarray*}
Since $\mu_1=\mu_0+\beta$, we express the likelihood in terms of two groups as follows
$$\log \ell(\mu_0,\mu_1)=-\log(2\pi\sigma^2)\left(\frac{n_0+n_1}{2}\right)
-\frac{1}{2\sigma^2}\sum_{i=1}^{n_0}(x_i-\mu_0)^2-\frac{1}{2\sigma^2}\sum_{i=1}^{n_1}(x_i-\mu_1)^2$$, 
where, $n_0$ corresponds to treatement $z_i=0$ and $n_1$ corresponds to treatment $z_i=1$ and $n=n_0+n_1$. \\
Under $H_0$, we have that $\mu_0=\mu_1$, so the likelihood reduces to 
$$\log \ell(\mu_0)=-\log(2\pi\sigma^2)\left(\frac{n_0+n_1}{2}\right)
-\frac{1}{2\sigma^2}\sum_{i=1}^{n_0}(x_i-\mu_0)^2-\frac{1}{2\sigma^2}\sum_{i=1}^{n_1}(x_i-\mu_0)^2$$
The MLE of $\mu_0$ under $H_0$ is:
\begin{eqnarray*}
 \frac{\partial \log \ell(\mu_0)}{\partial \mu_0}&=& \frac{1}{\sigma^2}\sum_{i=1}^{n_0}(x_i-\mu_0)+\frac{1}{\sigma^2}\sum_{i=1}^{n_1}(x_i-\mu_0)=0\\
&\Rightarrow& \tilde{\mu_0}=\frac{n_0\bar{x}_0+n_1\bar{x}_1}{n_0+n_1}
\end{eqnarray*}
which is a maximum since $ \frac{\partial^2 \log \ell(\mu_0)}{\partial \mu_0^2} = -\frac{1}{\sigma^2}(n_0+n_1)<0$.\\
Replacing $\tilde{\mu}_0$ on the score function we obtain $$U_\beta(0)=\dfrac{n_1n_0(\bar{x}_1-\bar{x}_0)}{\sigma^2(n_0+n_1)}$$
Since $Var(\bar{x}_1-\bar{x}_0)=\frac{\sigma^2}{n_1}+\frac{\sigma^2}{n_0}$, the Fisher Information is 
$$I_\beta(0)=\dfrac{n_1n_0}{\sigma^2(n_0+n_1)}$$ 
Since MSE is an unbiased estimator of the variance of the linear model fitted in part (b), we have $\hat{\sigma}^2=1.098$ and obtain:
\begin{table}[H]
\centering
\begin{tabular}{rrr}
  \hline
 & $U_k(0)$ & $I_k(0)$ \\ 
  \hline
$W=1$ & 2.06 & 5.94 \\ 
$W=2$ & 4.83 & 5.23 \\ 
$W=3$ & 4.67 & 4.97 \\ 
$W=4$ & 6.04 & 6.34 \\ 
   \hline
\end{tabular}
\caption{Score Functions and Fisher Information for each stratum of $W$}
\end{table}

\item[(d)] A score test for $H_0: \beta = 0$ is given by:
$$R_n=\dfrac{[\sum_k U_k(0)]^2}{\sum_k I_k(0)}=13.785,$$ which follows a $F$ distribution
 with 1 and 95 degrees of freedom. The p-value associated is $0.000346$, same as in (b). Notice that the value of the \emph{F} test statistic is the same as the \emph{t} statistics from part (b) squared. We reject the null hypothesis with a 95\% confidence.
\end{itemize}

\noindent
\textbf{Problem 2} 
\begin{itemize}
\item[(a)] Following is the contingency table for this problem
\begin{table}[ht]
\centering
\begin{tabular}{rlcr}
  \hline
 & A & B & Total\\ 
  \hline
Alive &  27=$X_1$ &  36 & 63=$n_1$ \\ 
  Death &  22 &   8& 30=$n_0$ \\ 
\hline
 &  49=$m$ &   44& 93=$N$\\
   \hline
\end{tabular}
\caption{Ignoring Strata}
\end{table}
\newpage
The uncorrected Score Test Statistic (Pearson chi-squared) is given by 
$$R_n=\dfrac{(X_1 N-mn_1)^2N}{n_0n_1m(N-m)} = 7.572$$ 
with an associated p-value of $0.0059$. We conclude that there is a difference in overall mortality between treatments A and B. 

\item[(b)] The stratified Score Test is given by $$R_n=\dfrac{[\sum_k U_k(0)]^2}{\sum_k I_k(0)},$$
in this case $k=3$:

\begin{table}[ht]
\centering
\begin{tabular}{rrr}
  \hline
 & $U_k(0)$ & $I_k(0)$ \\ 
  \hline
Stratum 1 & -1.93 & 1.30 \\ 
Stratum 2 & -1.88 & 1.14 \\ 
Stratum 3 & -1.83 & 0.96 \\ 
   \hline
\end{tabular}
\caption{Score Functions and Fisher Information per Stratum}
\end{table}

Which gives a result of $R_n=9.356$ with an associated p-value of 
0.0022 and we conclude that there is a difference in overall mortality between treatments A and B even after
we adjust for the different Stratum.
\end{itemize}

\noindent
\textbf{Problem 3} 
\begin{itemize}
\item[(a)]
Below is the Kaplan-Meier estimate of survival $\hat{S}(t)$, for the two groups combined
% latex table generated in R 3.1.1 by xtable 1.7-4 package
% Thu Oct  1 10:15:39 2015
\begin{table}[ht]
\centering
\begin{tabular}{rrrrr}
  \hline
time & n.risk & n.event & n.censor & survival \\ 
  \hline
8 & 20 & 0 & 1 & 1.0000 \\ 
  9 & 19 & 1 & 0 & 0.9474 \\ 
  11 & 18 & 0 & 1 & 0.9474 \\ 
  12 & 17 & 1 & 0 & 0.8916 \\ 
  13 & 16 & 1 & 0 & 0.8359 \\ 
  14 & 15 & 2 & 0 & 0.7245 \\ 
  16 & 13 & 1 & 1 & 0.6687 \\ 
  18 & 11 & 0 & 1 & 0.6687 \\ 
  19 & 10 & 0 & 1 & 0.6687 \\ 
  22 & 9 & 0 & 1 & 0.6687 \\ 
  23 & 8 & 1 & 1 & 0.5851 \\ 
  24 & 6 & 1 & 0 & 0.4876 \\ 
  26 & 5 & 1 & 0 & 0.3901 \\ 
  28 & 4 & 1 & 0 & 0.2926 \\ 
  29 & 3 & 0 & 1 & 0.2926 \\ 
  30 & 2 & 1 & 0 & 0.1463 \\ 
  31 & 1 & 1 & 0 & 0.0000 \\ 
   \hline
\end{tabular}
\end{table}
\begin{figure}[H]
\begin{center}
\includegraphics[scale = 0.6]{Plot3a.pdf}
\end{center}
\end{figure}

\item[(b)]
Variance of Kaplan-Meier's estimate of the survival function is given by:
\[ \hat{\Var}(\hat{S(t)})= \{\hat{S}(t)\}^2 \sum_{j: t_j^* \leq t} \frac{d_j}{n_j(n_j - d_j)} \]
Below are variances and $95\%$ confidence intervals.
% latex table generated in R 3.1.1 by xtable 1.7-4 package
% Thu Oct  1 10:50:35 2015
\begin{table}[H]
\centering
\begin{tabular}{rrrrr}
  \hline
  time & $\hat{S}(t)$ & $\hat{\Var}(\hat{S(t)}$ & Lower & Upper \\ 
  \hline
  9 & 0.9474 & 0.0026 & 0.8521 & 1.0000 \\ 
  12 & 0.8916 & 0.0052 & 0.7604 & 1.0000 \\ 
  13 & 0.8359 & 0.0075 & 0.6821 & 1.0000 \\ 
  14 & 0.7245 & 0.0110 & 0.5452 & 0.9626 \\ 
  16 & 0.6687 & 0.0123 & 0.4834 & 0.9252 \\ 
  23 & 0.5851 & 0.0155 & 0.3856 & 0.8880 \\ 
  24 & 0.4876 & 0.0187 & 0.2814 & 0.8448 \\ 
  26 & 0.3901 & 0.0196 & 0.1931 & 0.7879 \\ 
  28 & 0.2926 & 0.0181 & 0.1187 & 0.7213 \\ 
  30 & 0.1463 & 0.0152 & 0.0280 & 0.7646 \\ 
  31 & 0.0000 &  &  &  \\ 
   \hline
\end{tabular}
\end{table}

\item[(c)]
Below is the log-rank test to assess equality of treatments: 
% latex table generated in R 3.1.1 by xtable 1.7-4 package
% Thu Oct  1 11:09:49 2015
\begin{table}[H]
\centering
\begin{tabular}{rrrrr}
  \hline
  & N & Observed & Expected & $\frac{U(0)^2}{\I(0)} = \frac{(O - E)^2}{V}$ \\ 
  \hline
  grp A & 10 & 6 & 8.2597 & 2.4071 \\ 
  grp B & 10 & 6 & 3.7403 & 2.4071 \\ 
   \hline
\end{tabular}
\end{table}
With p-value $0.12079$, we do not have sufficient evidence to reject the null hypothesis.\\

Below is the Gehan-Wilcoxon test to assess equality of treatments: 
% latex table generated in R 3.1.1 by xtable 1.7-4 package
% Thu Oct  1 11:20:21 2015
\begin{table}[H]
\centering
\begin{tabular}{rrrrrr}
  \hline
  & N & Observed & Expected & $\frac{(O - E)^2}{V}$ \\ 
  \hline
  grp A & 10 & 3 & 4.8135 & 3.8207 \\ 
  grp B & 10 & 5 & 2.9923 & 3.8207 \\ 
   \hline
\end{tabular}
\end{table}
With p-value 0.0506 we do not have sufficient evidence to reject the null hypothesis.

\end{itemize}
\end{document}
