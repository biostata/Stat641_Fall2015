\documentclass[11pt,a4paper]{article}

%%%%%%%%%%%%%%%%%%%%%%%%%%%%%%%%%%%%%%%%%%%%%%%%%%%%%%%%%%%%%%%%%%%%%%
%% Input header file 
%%%%%%%%%%%%%%%%%%%%%%%%%%%%%%%%%%%%%%%%%%%%%%%%%%%%%%%%%%%%%%%%%%%%%%
\input{HeaderfileTexDocs}

%%%%%%%%%%%%%%%%%%% To change the margins and stuff %%%%%%%%%%%%%%%%%%%
\geometry{left=0.9in, right=0.9in, top=1in, bottom=0.8in}
%\setlength{\voffset}{0.5in}
%\setlength{\hoffset}{-0.4in}
%\setlength{\textwidth}{7.6in}
%\setlength{\textheight}{10in}
%%%%%%%%%%%%%%%%%%%%%%%%%%%%%%%%%%%%%%%%%%%%%%%%%%%%%%%%%%%%%%%%%%%%%%%
\begin{document}

\title{Homework 1}
\author{Subhrangshu Nandi\\
  Stat 641; Fall 2015}
\date{September 17, 2015}
%\date{}

\maketitle

\noindent
\textbf{Problem 1} Assuming that all variables other than \emph{z} are normally distributed.\\
A significance level of $\alpha=0.05$ was assumed to test all the hypotheses:

\begin{enumerate}
\item[(a)] The hypothesis for the two-sample t-test for differences in variable \emph{w} between treatment 
groups will be
\begin{eqnarray*}
  H_0 &:& \mu_{w_0} =\mu_{w_1}\\
  H_1 &:& \mu_{w_0} \neq \mu_{w_1},
\end{eqnarray*}
where $\mu_{w_0}$ and $\mu_{w_1}$ represent the mean of the variable \emph{w} for each treatment group (\emph{z} = 0, 1, respectively).\\
Assuming equal variances:
\begin{table}[h!]
\centering
\begin{tabular}{lll}
\hline
t-statistic & df & p-value\\
\hline
0.0189 & 98 & 0.985\\
\hline
\end{tabular}
\caption{Two sample t-test for \emph{w}}
\end{table}
There is no statistical evidence to reject the null hypothesis and we conclude that there is no significant difference 
in the means of the baseline variable \emph{w} for the two treatment groups.  

\item[(b)] For response variable \emph{x}:

\begin{table}[h!]
\centering
\begin{tabular}{ccc}
\hline
Treatment Group & Mean & Std. Dev.\\
\hline
 0 &13.944&  1.091\\
 1 &14.336 & 0.986\\
\hline
\end{tabular}
\caption{Mean and Std. Dev. for each treatment group}
\end{table}

Both standard deviations are close to one and the means of the response variable \emph{x} are similar but we would have to test it 
to make any statistical conclusion.

\item[(c)] The hypothesis for the two-sample t-test for differences in variable \emph{x} between treatment 
groups will be
\begin{eqnarray*}
  H_0 &:& \mu_{x_0} =\mu_{x_1}\\
  H_1 &:& \mu_{x_0} \neq \mu_{x_1},
\end{eqnarray*}
where $\mu_{x_0}$ and $\mu_{x_1}$ represent the mean of the variable \emph{x} for each treatment group (\emph{z}=0,1, respectively).\\
Assuming equal variances:

\begin{table}[h!]
\centering
\begin{tabular}{lll}
\hline
t-statistic & df & p-value\\
\hline
 -1.885 & 98 & 0.0624\\
\hline
\end{tabular}
\caption{Two sample t-test for \emph{x}}
\end{table}

There is no significant evidence to reject the null hypothesis and we conclude that the means between the two groups are equal. 

\newpage
\item[(d)] The plot is:

\begin{figure}[h!]
\centering
\includegraphics[scale=0.65]{Plot_d.pdf}
\vspace{-0.6cm}
\caption{\emph{x} vs \emph{w}}
\end{figure}

Figure 1 shows a linear pattern for both groups. No clear differences between them but the slope for $z=0$ seems 
that will be slightly steeper then the one for $z=1$.

\item[(e)] The result of the fitted model is:

\begin{table}[h!]
\centering
\begin{tabular}{lcccc}
  \hline
 & Estimate & Std. Error & t value & p-value \\ 
  \hline
Intercept & 11.7317 & 0.2223 & 52.78 & $\sim 0$ \\ 
 \textbf{z1} & 0.3949 & 0.1388 & 2.85 & \textbf{0.0054} \\ 
  w & 0.7345 & 0.0662 & 11.09 & $\sim 0$ \\ 
   \hline
\end{tabular}
\caption{Summary of \texttt{x$\sim$z+w} }
\end{table}

Adjusting by the baseline value \emph{w}, there is significant evidence to reject the null hypothesis ($H_0:\mu_{x_0} =\mu_{x_1}$)
 and we conclude that the means for the response variable \emph{x} between both treatments are not equal (given \emph{w}). 


%\begin{figure}[h!]
%\centering
%\includegraphics[scale=0.65]{e.pdf}
%\vspace{-0.6cm}
%\caption{Awesome Image}
%\end{figure}

\item[(f)] The answers differ since in part (c) we are not controlling by the baseline variable. Once we do, the 
 conclusion changes from no significant difference in the means to a significant one.\\
Given that \emph{w} is in the model, there is an association between \emph{x} and \emph{z} since the 
mean between the different groups are not statistically equal. This leads us to believe that \emph{w} is not a confounder variable. If it were, then given that \emph{w} is in the model there would be
 no association between \emph{x} and \emph{z}. 


\newpage
\item[(g)]

The hypothesis for the two-sample t-test for differences in variable \emph{w} between treatment 
groups will be
\begin{eqnarray*}
  H_0 &:& \mu_{y_0} =\mu_{y_1}\\
  H_1 &:& \mu_{y_0} \neq \mu_{y_1},
\end{eqnarray*}
where $\mu_{y_0}$ and $\mu_{y_1}$ represent the mean of the variable \emph{y} for each treatment group (\emph{z}=0,1, respectively).\\
Assuming equal variances:

    \begin{table}[h!]
\centering
\begin{tabular}{lll}
\hline
t-statistic & df & p-value\\
\hline
-1.6885 & 98 & 0.09449\\
\hline
\end{tabular}
\caption{Two sample t-test for \emph{y}}
\end{table}

Based on the p-value we conclude that there is no sifnicant evidence to reject the null hypothesis so the 
means for the response variable \emph{y} between both treatments are equal. 

\item[(h)] To perform a two sample t-test for the response variable \emph{y} between treatment groups, adjusting for the baseline variabe 
\emph{w} is necessary to fit the following linear model:

\begin{table}[h!]
\centering
\begin{tabular}{lcccc}
  \hline
 & Estimate & Std. Error & t value & p-value \\ 
  \hline
Intercept & 33.6902 & 2.5730 & 13.09 & $\sim 0$ \\ 
  \textbf{z1} & 3.3416 & 1.6066 & 2.08 & \textbf{0.0402} \\ 
  w & 5.4083 & 0.7665 & 7.06 & $\sim 0$ \\ 
   \hline
\end{tabular}
\caption{Summary of \texttt{y$\sim$z+w} }
\end{table}

We can see that the conclusion changes adjusting for the baseline. In this case, there is significant evidence to 
reject the null hypothesis ($p-value=0.0402<\alpha=0.05$) and we conclude that the 
means for the response variable \emph{y} between both treatments are not equal, adjusted by \emph{w}.


\item[(i)] The result of the fitted model is:

\begin{table}[h!]
\centering
\begin{tabular}{lcccc}
  \hline
 & Estimate & Std. Error & t value & p-value \\ 
  \hline
Intercept & -63.6200 & 9.8354 & -6.47 & $\sim 0$ \\ 
 \textbf{z1} & 0.0658 & 1.1726 & 0.06 & \textbf{0.9554} \\ 
  w & -0.6841 & 0.8095 & -0.85 & 0.4002 \\ 
  x & 8.2946 & 0.8241 & 10.06 & $\sim 0$ \\ 
   \hline
\end{tabular}
\caption{Summary of \texttt{y$\sim$z+w+x} }
\end{table}

Adjusting by both \emph{w} and \emph{x}, there is no significant evidence to reject the null hypothesis ($H_0:\mu_{y_0} =\mu_{y_1}$)
 and we conclude that the means for the response variable \emph{y} between both treatments are equal (given \emph{w} and \emph{x}). 

\newpage
\item[(j)] The results from the three previous questions can be summarized as:

\begin{table}[h!]
\centering
\begin{tabular}{clcc}
\hline
 & Model & p-value for \emph{z} & Diff. bewtween \emph{y} and \emph{z}\\
\hline
(g) & \texttt{y$\sim$z} & 0.0945 & NO\\
(h) & \texttt{y$\sim$z+w} & 0.0402 & YES\\
(i) & \texttt{y$\sim$z+w+x} & 0.9554 & NO\\
\hline
\end{tabular}
\caption{Results for (g), (h) and (i)}
\end{table}

As we can see, the conclusions change given which variables are in the model. In the models from part 
(g) and (i), simple two sample t-test and adjusting by \emph{w} and \emph{x}, we reach the same conclusion: there is no significant difference between the means but in the fitted model
 for part (h), adjusting only by \emph{w}, we conclude that there is a significant difference between the mean for both treatments. \\
Adjusting by \emph{w}, once \emph{x} is in the model, there is no difference in the mean of
 \emph{y} between treatment groups (no association). When we don't include \emph{x} in the model, we may be
 reaching the wrong conclusion since we can be stating a false association between the two variables. This leads 
us to think that \emph{x} can be a confounder variable for the association between \emph{z} 
and \emph{y} (considering the baseline in the model).

\end{enumerate}

\end{document}
