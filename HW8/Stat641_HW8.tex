\documentclass[11pt,a4paper]{article}

%%%%%%%%%%%%%%%%%%%%%%%%%%%%%%%%%%%%%%%%%%%%%%%%%%%%%%%%%%%%%%%%%%%%%%
%% Input header file 
%%%%%%%%%%%%%%%%%%%%%%%%%%%%%%%%%%%%%%%%%%%%%%%%%%%%%%%%%%%%%%%%%%%%%%
\input{HeaderfileTexDocs}

%%%%%%%%%%%%%%%%%%% To change the margins and stuff %%%%%%%%%%%%%%%%%%%
\geometry{left=0.9in, right=0.9in, top=0.9in, bottom=0.8in}
%\setlength{\voffset}{0.5in}
%\setlength{\hoffset}{-0.4in}
%\setlength{\textwidth}{7.6in}
%\setlength{\textheight}{10in}
%%%%%%%%%%%%%%%%%%%%%%%%%%%%%%%%%%%%%%%%%%%%%%%%%%%%%%%%%%%%%%%%%%%%%%%
\begin{document}

\title{Homework 8}
\author{Subhrangshu Nandi\\
  Stat 641; Fall 2015}
\date{December 10, 2015}
%\date{}

\maketitle

\begin{enumerate}
\item[(a)] At interim analysis period 1:\\
$n_0 = 135, n_1 = 145, W = 20792$. Since both $n_1$ and $n_0$ are large enough, $W \sim \mathcal{N}(\mu,\sigma^2)$, where, 
\[ \mu = \frac{n_1(n_1 + n_0 + 1)}{2} = 20372.5, \ \ \sigma^2 = \frac{n_0 n_1(n_1 + n_0 + 1)}{12} = 458481.2\] 
Testing the hypothesis: 
\[ H_0: \mu_1 = \mu_0\ \text{  vs  } \ H_a: \mu_1 > \mu_0\]
the normal approximation yields a p-value of 0.2678, and the Wilcoxon test gives p-value 0.268. \\
\noindent
At interim analysis period 2:\\
$n_0 = 336, n_1 = 320, W = 110400, \mu = 105120, \sigma = 2426.26$. The normal approximation and the Wilcoxon test both yield a p-value of 0.0148. \\
\noindent
At final analysis period 3:\\
$n_0 = 522, n_1 = 478, W = 247590, \mu = 239239, \sigma = 4562.21$. The normal approximation and the Wilcoxon test both yield a p-value of 0.0336. \\
So, we reject the null hypothesis in favor of the alternative, $\mu_1 > \mu_0$
\item[(b)] After generating 10,000 random samples of $\mathbf{T} = (T_1 - \Exp(T_1), T_2 - \Exp(T_2), T_3 - \Exp(T_3))$, the covariance and correlation matrices are
% latex table generated in R 3.2.2 by xtable 1.7-4 package
% Wed Dec  9 13:42:11 2015
\begin{table}[H]
\centering
\begin{tabular}{rrrr}
  \hline
& $T_1 - \Exp(T_1)$ & $T_2 - \Exp(T_2)$ & $T_3 - \Exp(T_3)$ \\ 
  \hline
$T_1 - \Exp(T_1)$ & 455507 & 20095 & 50488 \\ 
$T_2 - \Exp(T_2)$ & 20095 & 5952464 & -226825 \\ 
$T_3 - \Exp(T_3)$ & 50488 & -226825 & 21361762 \\ 
  \hline
\end{tabular}
\caption{Covariance matrix of $\mathbf{T}$}
\end{table}
% latex table generated in R 3.2.2 by xtable 1.7-4 package
% Wed Dec  9 13:26:50 2015
\begin{table}[H]
\centering
\begin{tabular}{rrrr}
\hline
& $T_1 - \Exp(T_1)$ & $T_2 - \Exp(T_2)$ & $T_3 - \Exp(T_3)$ \\ 
\hline
  $T_1 - \Exp(T_1)$ & 1.0000 & 0.0122 & 0.0162 \\ 
  $T_2 - \Exp(T_2)$ & 0.0122 & 1.0000 & -0.0067 \\ 
  $T_3 - \Exp(T_3)$ & 0.0162 & -0.0067 & 1.0000 \\ 
\hline
\end{tabular}
\caption{Correlation matrix of $\mathbf{T}$}
\end{table}
Since the correlation terms are so small, it can be concluded that the sequence $T_1 - \Exp(T_1), T_2 - \Exp(T_2), T_3 - \Exp(T_3)$ are independent.

\item[(c)] After generating 10,000 random samples of $\mathbf{S} = (S_1 - \Exp(S_1), S_2 - \Exp(S_2), S_3 - \Exp(S_3))$, the covariance and correlation matrices are
% latex table generated in R 3.2.2 by xtable 1.7-4 package
% Wed Dec  9 13:48:44 2015
\begin{table}[H]
\centering
\begin{tabular}{rrrr}
  \hline
& $S_1 - \Exp(S_1)$ & $S_2 - \Exp(S_2)$ & $S_3 - \Exp(S_3)$ \\   \hline
  \hline
  $S_1 - \Exp(S_1)$ & 2 24.98 & 0.13 & 0.32 \\ 
  $S_2 - \Exp(S_2)$ & 0.13 & 57.54 & -0.44 \\ 
  $S_3 - \Exp(S_3)$ & 0.32 & -0.44 & 75.97 \\ 
  \hline
\end{tabular}
\caption{Covariance matrix of $\mathbf{S}$}
\end{table}
% latex table generated in R 3.2.2 by xtable 1.7-4 package
% Wed Dec  9 13:36:33 2015
\begin{table}[H]
\centering
\begin{tabular}{rrrr}
  \hline
& $S_1 - \Exp(S_1)$ & $S_2 - \Exp(S_2)$ & $S_3 - \Exp(S_3)$ \\   \hline
  $S_1 - \Exp(S_1)$ & 1.0000 &  0.0034 & 0.0074 \\ 
  $S_2 - \Exp(S_2)$ & 0.0034 & 1.0000 & -0.0067 \\ 
  $S_3 - \Exp(S_3)$ & 0.0074 & -0.0067 & 1.0000 \\ 
   \hline
\end{tabular}
\caption{Correlation matrix of $\mathbf{S}$}
\end{table}
Since the correlation terms are even smaller than $Cov(\mathbf{T})$, it can be concluded that the sequence $S_1 - \Exp(S_1), S_2 - \Exp(S_2), S_3 - \Exp(S_3)$ are independent.

\item[(d)] The variances of $\mathbf{S}$ from the theoretical distribution are 25.15, 57.49 and 76.39. From the randomization distribution, the variances are 24.98, 57.54 and 75.97 (very close). \\ 
The information fractions using the theoretical (expected) distributions are: 32.93\% and 75.26\%. \\
The information fractions using the randomized distributions are: 32.88\% and 75.74\%. 

\item[(e)] Using the $\alpha$-spending function $g(t) = 0.025t$, the Lan-DeMets bounds are:
% latex table generated in R 3.2.2 by xtable 1.7-4 package
% Wed Dec  9 14:41:05 2015
\begin{table}[H]
\centering
\begin{tabular}{rrrrrr}
  \hline
  Time & Lower & Upper & Exit pr. & Diff. pr. & Nominal Alpha \\ 
  \hline
  0.3293 & -2.6424 & 2.6424 & 0.0082 & 0.0082 & 0.0082 \\ 
  0.7526 & -2.4959 & 2.4959 & 0.0188 & 0.0106 & 0.0126 \\ 
  1.0000 & -2.5096 & 2.5096 & 0.0250 & 0.0062 & 0.0121 \\ 
   \hline
\end{tabular}
\caption{Lan-DeMets bounds for $g(t) = 0.026t$, theoretical distribution}
\end{table}
The critical values at the information fractions derived from the previous part using theoretical variances are 2.6424, 2.4959 and 2.5096. \\
Using the randomization distribution of the Z-statistic derived from $\mathbf{S}$, the probability of stopping at the two interim analysis are 0.0041 and 0.0104 respectively and the probability of rejecting $H_0$ at the final analysis is 0.0164. These are lower than the theoretical values based on $g(t)$

\item[(f)] Using the information fractions from the randomized distributions, the Lan-DeMets bounds are:
% latex table generated in R 3.2.2 by xtable 1.7-4 package
% Wed Dec  9 21:51:35 2015
\begin{table}[H]
\centering
\begin{tabular}{rrrrrr}
  \hline
  Time & Lower & Upper & Exit pr. & Diff. pr. & Nominal Alpha \\ 
  \hline
  0.3288 & -2.6429 & 2.6429 & 0.0082 & 0.0082 & 0.0082 \\ 
  0.7574 & -2.4925 & 2.4925 & 0.0189 & 0.0107 & 0.0127 \\ 
  1.0000 & -2.5110 & 2.5110 & 0.0250 & 0.0061 & 0.0120 \\ 
   \hline
\end{tabular}
\caption{Lan-DeMets bounds for $g(t) = 0.026t$, randomized distribution}
\end{table}
The critical values at the information fractions derived from the previous part using theoretical variances are 2.6429, 2.4925 and 2.5110. \\ This procedure will reject $H_0$ at the second stage.

\end{enumerate}

\end{document}


