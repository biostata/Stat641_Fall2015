\documentclass[11pt,a4paper]{article}

%%%%%%%%%%%%%%%%%%%%%%%%%%%%%%%%%%%%%%%%%%%%%%%%%%%%%%%%%%%%%%%%%%%%%%
%% Input header file 
%%%%%%%%%%%%%%%%%%%%%%%%%%%%%%%%%%%%%%%%%%%%%%%%%%%%%%%%%%%%%%%%%%%%%%
../../../TexScripts/HeaderfileTexDocs.tex

%%%%%%%%%%%%%%%%%%% To change the margins and stuff %%%%%%%%%%%%%%%%%%%
\geometry{left=0.9in, right=0.9in, top=1in, bottom=0.8in}
%\setlength{\voffset}{0.5in}
%\setlength{\hoffset}{-0.4in}
%\setlength{\textwidth}{7.6in}
%\setlength{\textheight}{10in}
%%%%%%%%%%%%%%%%%%%%%%%%%%%%%%%%%%%%%%%%%%%%%%%%%%%%%%%%%%%%%%%%%%%%%%%
\begin{document}

\title{Homework 5}
\author{Subhrangshu Nandi\\
  Stat 641; Fall 2015}
\date{November 5, 2015}
%\date{}

\maketitle

\noindent
\textbf{(1)} 
We observe the following:

\begin{table}[ht]
\centering
\begin{tabular}{lcccc}
  \hline
 & DFS & Recurrence & Dead & Total \\ 
  \hline
	Treatment 1 & 33 & 17 & 41 & 91\\
	Treatment 2 & 44 & 18 & 25 & 87\\
   \hline
	Total  & 77 & 35 & 66 & 178\\
   \hline
\end{tabular}
\caption{Results}
\end{table}
\begin{enumerate}
\item[(a)] Mean Ranks: \\
DRS: $\frac{1 + 77}{2} = \frac{78}{2} = 39$ \\
Recurrence: $\frac{78 + 112}{2} = \frac{190}{2} = 95$\\
Dead: $\frac{113+178}{2} = \frac{291}{2} = 145.5$
Rank sum in group 2: $$T_2=44\cdot\frac{78}{2}+18\cdot\frac{190}{2}+25\cdot\frac{291}{2}=7063.5$$
Expected rank sum$=87\cdot\frac{179}{2}=7786.5$
$$T_2-E(T_2)=7063.5-7786.5=-723$$
Variance of $T_2$: 
$$\frac{91\cdot 87}{178-1}\left[\frac{1}{178}\left\{ 77\left(\frac{78}{2}\right)^2 +35\left(\frac{190}{2}\right)^2 +66\left(\frac{291}{2}\right)^2\right\} - \frac{(178+1)^2}{4}\right]=101620.8 $$

The test statistic is $$\frac{(T_2-E(T_2))^2}{\Var(T_2)} = \frac{(-723)^2}{101620.8} = 5.144$$ 
This statistic has (asymptotically) a chi-square distribution with 1 df, and has a p-value of 0.023. 

\item[(b)]
Reject the null hypothesis of equality of treatments. Lower average rank in group 2 suggests that subjects in treatment group 2 tend to have
better outcomes than those in treatment group 1, so treatment 2 is superior to treatment 1. 

\end{enumerate}
\noindent
\textbf{(2)} 
\begin{enumerate}
\item[(a)]
\begin{table}[H]
\centering
\begin{tabular}{lcccccccccccccccc}
\hline
Obs 	& 0.2 & 0.8 & 1.9 & 2.2 & 2.6 & 2.8 & 3.9 & 5.1 & 7.1 & 7.7 & 8.2 & 12.3 & 18.8 & 21.8 & 27.1 & 39.7 \\
Rank  	& 1 & 2 & 3 & 4 & 5 & 6 & 7 & 8 & 9 & 10 & 11 & 12 & 13 & 14 & 15 & 16 \\ 
Group	& C & C & C & C & C & E & C & E & E & E & C & E & E & C & E & E \\
\hline
\end{tabular}
\end{table}
For the Wilcoxon rank sum test statistic we need:
\begin{eqnarray*}
T_B &=&\sum_{i=1}^{n_B} R_{iB}=89\\
E(T_B)&=&n_B\frac{n_A+n_B+1}{2}=68\\
Var(T_B)&=& \frac{n_An_B(n_A+n_B+1)}{12}=90.667
\end{eqnarray*}
With this we get the test statistic $\frac{(T_2-E(T_2))^2}{Var(T_2)}=4.8639$, which is asymptotically chi-squared distributed, which yields a p-value of 0.02742. 
\item[(b)] For each observation in group ``E'', $M_j$ is the number of ``C'' subjects that are smaller:
\begin{table}[H]
\centering
\begin{tabular}{lcccccccc}
\hline
Experimental 	& 2.8 & 5.1 & 7.1 & 7.7 & 12.3 & 18.8 & 27.1 & 39.7 \\
$M_j$  		& 5 & 6 & 6 & 6 & 7 & 7 & 8 & 8 \\ 
\hline
\end{tabular}
\end{table}
The sum of the values in the second row is $5 + 6 + 6 + 6 + 7 + 7 + 8 + 8 = 53$. The expected value is $8 × 8/2 = 32$. Hence $U = 53 − 32 = 21$

\item[(c)] T-test for comparing the two groups:
\begin{table}[H]
\centering
\begin{tabular}{lcc}
  \hline
t-statistic & df (aprox.) & p-value\\
\hline
-1.9093 & 10.993 & 0.08265\\
   \hline
\end{tabular}
\caption{t-test for unequal variances}
\end{table}

There is not enough evidence to reject the null hypothesis that the means between both groups are equal. 

\end{enumerate}
\noindent
\textbf{(3)} 
\begin{enumerate}
\item[(a)] The effect of treatment on mortality can be tested by Pearson's chi-square test on
% latex table generated in R 3.2.2 by xtable 1.7-4 package
% Thu Nov  5 11:33:43 2015
\begin{table}[H]
\centering
\begin{tabular}{rrr}
  \hline
 & Alive & Dead \\ 
  \hline
Trt0 &  88 &  12 \\ 
  Trt1 &  66 &  34 \\ 
   \hline
\end{tabular}
\end{table}
The test statistic is 13.665, with df 1, and p-value 0.0002. We reject the null hypothesis of no dependence. 

\item[(b)] To examine the difference between treatment groups for the the survivors:
\begin{figure}[H]
\begin{center}
\includegraphics[scale=0.4]{Plotb.pdf}
\end{center}
\end{figure}
A t-test does not reveal significant difference between the two treatment groups, conditioned on survival:
% latex table generated in R 3.2.2 by xtable 1.7-4 package
% Thu Nov  5 11:59:08 2015
\begin{table}[H]
\centering
\begin{tabular}{rrrrr}
  \hline
 & Estimate & Std. Error & t value & Pr($>$$|$t$|$) \\ 
  \hline
(Intercept) & 10.5898 & 0.3422 & 30.94 & 0.0000 \\ 
  zTrt1 & -0.5549 & 0.5228 & -1.06 & 0.2901 \\ 
   \hline
\end{tabular}
\end{table}
This is conditioned on survival, which is a post-randomization phenomenon. So, this cannot be considered as a causal effect of treatment, even if effect was significant.
\item[(c)]
Replacing the 'NAs' with 2 in the variable $y$, the boxplot for assessing the difference between treatments:
\begin{figure}[H]
\begin{center}
\includegraphics[scale=0.4]{Plotc.pdf}
\end{center}
\end{figure}
Below is a model for assessing the difference between treatments
% latex table generated in R 3.2.2 by xtable 1.7-4 package
% Thu Nov  5 12:06:21 2015
\begin{table}[H]
\centering
\begin{tabular}{rrrrr}
  \hline
 & Estimate & Std. Error & t value & Pr($>$$|$t$|$) \\ 
  \hline
(Intercept) & 9.5590 & 0.4378 & 21.84 & 0.0000 \\ 
  zTrt1 & -2.2560 & 0.6191 & -3.64 & 0.0003 \\ 
   \hline
\end{tabular}
\end{table}

\item[(d)]
(i) Difference in Mean scores is 2.256. The mean of group with Trt1 is lower than the group with Trt0. \\

(ii) Median score of Trt0 is 9.5 and that of Trt1 is 8.1. The difference in median is 1.4 \\

(iii) Hodges-Lehmann estimate is approximately 2.5 \\
The mean difference of scores does not make sense because the response values for the ``dead'' people have been imputed. The median difference in response is more meaningful.
\item[(e)] Death is a competing risk for non-fatal events. Once someone dies without a prior non-fatal event, they are no longer at risk for non-fatal events. Death may be associated with both treatment and non-fatal event process. The observed non-fatal event process is confounded by death. As ``censoring'', it is informative (not independent). It is impossible to statistically separate effects of treatment on non-fatal events from its effect on death (non-identifiability).

\end{enumerate}
\noindent
\textbf{(4)} 
The hypothesis of interest are $H_0:p\leq 0.15$ versus $H_1:p\geq 0.4$ where $p$ is the true success rate. 
\begin{enumerate}
\item[(a)] Under $H_0$: $Y_1\sim Bin (16,0.15)$, the stopping probability will be $$P(Y_1\leq 3)=0.7899.$$
Under $H_1$: $Y_1\sim Bin (16,0.4)$, the stopping probability will be $$P(Y_1\leq 3)=0.0651.$$
\item[(b)] We can think that $P(\text{reject }H_0)=1-P(\text{accept }H_0)$.\\
Under $H_0$: $Y_1\sim Bin (16,0.15)$ and $Y_2\sim Bin (16,0.15)$:
\begin{eqnarray*}
P(\text{acept }H_0) &=& P(Y_1\leq 3)+P(3<Y_1<9,Y_1+Y_2\leq 8)\\
&=& \sum_{i=0}^{3} P(Y_1=i)+\sum_{i=4}^{8}\left[P(Y_1=i)\sum_{j=0}^{8-i}P(Y_2=j)\right]\\
&=& 0.7899 + 0.176\\
&=& 0.9659
\end{eqnarray*}
So, $$P(\text{reject }H_0)=1-0.9659=0.0341.$$
Under $H_1$: $Y_1\sim Bin (16,0.4)$ and $Y_2\sim Bin (16,0.4)$:
\begin{eqnarray*}
P(\text{acept }H_0) &=& P(Y_1\leq 3)+P(3<Y_1<9,Y_1+Y_2\leq 8)\\
&=& \sum_{i=0}^{3} P(Y_1=i)+\sum_{i=4}^{8}\left[P(Y_1=i)\sum_{j=0}^{8-i}P(Y_2=j)\right]\\
&=& 0.0651+0.0318\\
&=& 0.0969
\end{eqnarray*}
So, $$P(\text{reject }H_0)=1-0.0969=0.9031.$$

 
\item[(c)] The expected sample size can be derived by the fact that 
\begin{eqnarray*}
E(N)&=&m_1 P(\text{stopping at stage 1})+(m_1+m_2)P(\text{not stopping at stage 1})\\
&=& m_1 P(\text{stopping at stage 1})+(m_1+m_2)(1-P(\text{stopping at stage 1}))\\
&=& m_1+m_2-m_2P(\text{stopping at stage 1})
\end{eqnarray*}
For $p=0.15$ then $P(\text{stopping at stage 1})=0.7899$ 
so the expected sample size is\\ $E(N)=32-16\cdot 0.7899=19.3616\approx 20$.\\
For $p=0.15$ then $P(\text{stopping at stage 1})=0.7899$ 
so the expected sample size is\\ $E(N)=32-16\cdot 0.0651=30.9584\approx 31$.\\
\item[(d)] We can define $Y\sim Binom (32,p)$.\\
The type I error rate will be 
\begin{eqnarray*}
P(\text{reject }H_0|H_0 \text{ is true})&=& P(Y>8|p=0.15)\\
&=& 1-P(Y\leq 8|p=0.15) \\
&=& 0.0413
\end{eqnarray*}
The type II error rate will be 
\begin{eqnarray*}
P(\text{not reject }H_0|H_1 \text{ is true})&=& P(Y\leq 8|p=0.4)\\
&=& 0.0575
\end{eqnarray*}
The advantage of the two stage trial is that gives us the option to (possibly) treat fewer patients since we can stop in stage 1 if the null hypothesis is true.  
\end{enumerate}

\end{document}
