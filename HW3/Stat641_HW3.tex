\documentclass[11pt,a4paper]{article}

%%%%%%%%%%%%%%%%%%%%%%%%%%%%%%%%%%%%%%%%%%%%%%%%%%%%%%%%%%%%%%%%%%%%%%
%% Input header file 
%%%%%%%%%%%%%%%%%%%%%%%%%%%%%%%%%%%%%%%%%%%%%%%%%%%%%%%%%%%%%%%%%%%%%%
../../../TexScripts/HeaderfileTexDocs.tex

%%%%%%%%%%%%%%%%%%% To change the margins and stuff %%%%%%%%%%%%%%%%%%%
\geometry{left=0.9in, right=0.9in, top=1in, bottom=0.8in}
%\setlength{\voffset}{0.5in}
%\setlength{\hoffset}{-0.4in}
%\setlength{\textwidth}{7.6in}
%\setlength{\textheight}{10in}
%%%%%%%%%%%%%%%%%%%%%%%%%%%%%%%%%%%%%%%%%%%%%%%%%%%%%%%%%%%%%%%%%%%%%%%
\begin{document}

\title{Homework 3}
\author{Subhrangshu Nandi\\
  Stat 641; Fall 2015}
\date{October 8, 2015}
%\date{}

\maketitle

\noindent
\textbf{(a)} 
\begin{figure}[H]
\centering
\includegraphics[scale=0.65]{Plota.pdf}
\vspace{-0.5cm}
\caption{Kaplan-Meier estimates by treatment group}
\end{figure}

\noindent
\textbf{(b)} 
Under proportional hazards, $$\log \Lambda (t)=\log \Lambda_0 (t)+\beta \text{ trt}.$$
Testing that there is no difference in survival by treatment is equivalent to test $H_0:\beta=0$:

\begin{table}[H]
\centering
\begin{tabular}{lcc}
  \hline
Test &Test Statistic & p-value \\
\hline
Wald&5.86&0.01552\\
Score&5.91&0.01507\\
Likelihood Ratio & 5.98  &0.01445 \\
   \hline
\end{tabular}
\caption{Summary of tests}
\end{table}

For the three different test we reach the same conclusion: reject the null hypothesis with a 5\% sigificance and conclude that there is difference in the survival times between the two treatment groups.

\noindent
\textbf{(c)} 
From the fitted model we get:

\begin{table}[H]
\centering
\begin{tabular}{lccc}
  \hline
    &  $\hat{\beta}$ & $e^{\hat{\beta}}$ &$SE(\hat{\beta})$ \\ 
   \hline 
trt &-0.3231 &   0.7239 &  0.1335 \\
   \hline
\end{tabular}
\caption{Summary of treatment difference}
\end{table}
The 95\% confidence interval is $(0.5573, 0.9404)$.

\noindent
\textbf{(d)} If we add \emph{age} and \emph{sex} to the model:
\begin{table}[H]
\centering
\begin{tabular}{lcc}
  \hline
Test &Test Statistic & p-value \\
\hline
Wald& 14.23 & 0.002613 \\
Likelihood Ratio&  14.54 & 0.002257 \\
\hline
\end{tabular}
\caption{Summary of tests adjusting for \emph{age} and \emph{sex}}
\end{table}

For both tests we reject the null hypothesis with a 5\% significance and conclude that there is difference in the survival times between the two treatment groups, even after adjusting for \emph{age} and \emph{sex}.

\noindent
\textbf{(e)} 

\begin{figure}[H]
\centering
\includegraphics[scale=0.65]{Plote.pdf}
\vspace{-0.5cm}
\caption{Log cumulative hazard versus log time}
\end{figure}
These curves remain roughly the same distance apart for the portion where they are most stable; there is no evidence from the plot that the PH assumption does not hold.

We also perform a test for the Cox regression model that was fitted in part (d):
\begin{table}[H]
\centering
\begin{tabular}{lccc}
  \hline
& rho & chisq & p-value \\
\hline
trt             &0.0296 &0.20868 &0.648\\
sex2 &0.0154 &0.05721 &0.811\\
age             &0.0024 &0.00132 &0.971\\
GLOBAL          &    NA &0.27726 &0.964\\
   \hline
\end{tabular}
\caption{Proportional hazards tests results}
\end{table}
There is no evidence that PH assumption is violated

\noindent
\textbf{(f)} 
{\emph{Age}} is not a confounder because even after controlling for age, the coefficient and standard error of $\beta_{\text{trt}}$ does not change much.
\end{document}
