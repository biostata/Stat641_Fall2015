\documentclass[11pt,a4paper]{article}

%%%%%%%%%%%%%%%%%%%%%%%%%%%%%%%%%%%%%%%%%%%%%%%%%%%%%%%%%%%%%%%%%%%%%%
%% Input header file 
%%%%%%%%%%%%%%%%%%%%%%%%%%%%%%%%%%%%%%%%%%%%%%%%%%%%%%%%%%%%%%%%%%%%%%
../../../TexScripts/HeaderfileTexDocs.tex

%%%%%%%%%%%%%%%%%%% To change the margins and stuff %%%%%%%%%%%%%%%%%%%
\geometry{left=0.9in, right=0.9in, top=1in, bottom=0.8in}
%\setlength{\voffset}{0.5in}
%\setlength{\hoffset}{-0.4in}
%\setlength{\textwidth}{7.6in}
%\setlength{\textheight}{10in}
%%%%%%%%%%%%%%%%%%%%%%%%%%%%%%%%%%%%%%%%%%%%%%%%%%%%%%%%%%%%%%%%%%%%%%%
\begin{document}

\title{Article Summary 2 \\ Influencing healthful food choices in school and home environments:
results from the TEENS study}
\author{Subhrangshu Nandi\\
  Stat 641; Fall 2015}
\date{October 29, 2015}
%\date{}

\maketitle
% --------------------------------------------------------------
%                         Start here
% --------------------------------------------------------------

%\noindent
\textbf{Background} \\
The study aim is to investigate the effects of an intervention procedure on the increased access to healthful foods both at home and at school. As a  healthful diet is a key component of good and sustained health, it could be beneficial to provide increased access to healthful foods. \\

%\noindent
\textbf{Study Design} \\
Sixteen schools were selected for the TEEN study.
The experimental units in this trial are the schools. Within matched pairs, schools were randomly assigned either the control or intervention.
The intervention of interest included sending to families newsletters that included behavioral  coupons and advisory councils who worked with the school food service to promote increased availability of healthy a la carte items in the lunchroom. Schools which received the intervention were adviced by the SNAC councils. It is not clear how actual families were selected for either control or treatment. It is not clear if the family of a child enrolled in a school selected for the intervention would be given an intervention. While sixteen schools were selected and 3600 children were enrolled in the eight schools selected for intervention, it is not clear how many families were selected for intervention and how many were selected for control. 
A random subset of families (526 in total) received post-study surveys that measured which of a list of 43 food items were in the family home at the time of the survery. The families were also asked to compare purchase preference of paired items. At the school level, the availability of healthy food items at the schools were measured at eight time points (baseline, interim, and follow-up times). Also measured were the total number of students who purchased healthy lunches and the types of healthy items sold. Healthy a la carte items available were also measured over a period of five days. 

For the parent survery, a mixed model was used to analyze the results, but the specific model used and the analysis procedure are notably missing from the paper. It is unclear what variables were included in the model. School was modeled as a random effect and the reason for doing so is not mentioned in the paper. It is not totally clear that this is appropriate.

For the school level interventions, it is unclear what model was used other than the vague statement that they used an ANCOVA model. The outcome of interest in this analysis was the proportion of items offered and sold from the "Foods to Limit" and "Foods to Promote" categories. \\

%\noindent
\textbf{Efficacy Results} \\
There was no significant difference between intervention group and control group in home food item inventories. Parents in the intervention group were more likely to prefer a healthier item from a paired list. There is a weak statistical evidence of this effect. For the school level portion, no significant intervention effects were observed for meal plan offerings. There is weak evidence that intervention schools offered more healthy a la carte items than control schools. \\

%\noindent
\textbf{Discussion} \\
The presentation of the methods, goals, study design, and analysis plan are unclear. The study design was clearly inappropriate for the goals at hand. For school level effects, they only had a sample size of 16. For family level effects, they only measured a small random subset of the groups and had a large amount of missing responses. Their measurement for the family level effects was limited in that they only measured at one specific day whether a family had items in a list of 43 healthy items. There is much variability of home food inventories and simply measuring the presence of a small list of healthy items is not a good indication of whether an intervention actually increased healthful food availability in a home. It is also not clear if the healthy items at schools were priced competitively. If healthful food items were offered, but were more expensive, this would counteract the specific aims of the study. 
 
\end{document}
