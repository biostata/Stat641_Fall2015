\documentclass[11pt,a4paper]{article}

%%%%%%%%%%%%%%%%%%%%%%%%%%%%%%%%%%%%%%%%%%%%%%%%%%%%%%%%%%%%%%%%%%%%%%
%% Input header file 
%%%%%%%%%%%%%%%%%%%%%%%%%%%%%%%%%%%%%%%%%%%%%%%%%%%%%%%%%%%%%%%%%%%%%%
../../../TexScripts/HeaderfileTexDocs.tex

%%%%%%%%%%%%%%%%%%% To change the margins and stuff %%%%%%%%%%%%%%%%%%%
\geometry{left=0.9in, right=0.9in, top=1in, bottom=0.8in}
%\setlength{\voffset}{0.5in}
%\setlength{\hoffset}{-0.4in}
%\setlength{\textwidth}{7.6in}
%\setlength{\textheight}{10in}
%%%%%%%%%%%%%%%%%%%%%%%%%%%%%%%%%%%%%%%%%%%%%%%%%%%%%%%%%%%%%%%%%%%%%%%
\begin{document}

\title{Article Summary 3 \\ Effect of metoprolol CR/XL in chronic heart failure: Metoprolol
CR/XL Randomised Intervention Trial in Congestive Heart Failure
(MERIT-HF)
}
\author{Subhrangshu Nandi\\
  Stat 641; Fall 2015}
\date{December 1, 2015}
%\date{}

\maketitle
% --------------------------------------------------------------
%                         Start here
% --------------------------------------------------------------

%\noindent
\textbf{Background}
Chronic heart failure is a common disorder among the elderly which has a negative effect on mortality and quality of life. Metoprolol is a drug designed to improve cardiac function and lessen symptoms for heart failure. If such a drug is effective at also improving mortality, it could be useful in mitigating chronic heart failure. \\

\textbf{Study Design}
The targeted population for this study is men and women from ages 40-80 who have had symptomatic heart failure recently. Patients with lower than normal ejection fraction (less than 40\%) were included, as well as those with a few other heart conditions. Those excluded were those with severe complications including acute myocardial infarction, among a long list of conditions. The sample size for this study is 3991 patients, 1990 of which were assigned to treatment with metoprolol, and 2001 of which were assigned to placebo. The study was started with a single-blind placebo run-in and patients were randomized within groups according to ten factors. The motivation for such a randomization procedure is not mentioned. Patients were asked to come for follow-up visits every 3 months. A long list of conditions were monitored by an independent committee. The outcome measures were all-cause mortality and all-cause hospitalization. The analysis was performed in an intention-to-treat fashion. A Cox proportional hazards model was used to model relative risks of the treatment. \\

\textbf{Efficacy Results}
The study was terminated early by the recommendation of the independent safety committee. 145 patients in the treatment group died and 217 in the placebo group died. The all-cause mortality rate for the treatment group was significantly lower than that of placebo. Furthermore, there was also a reduction in cardiovascular mortality in the treatment group versus placebo. There is no information regarding the results for secondary outcomes, which were hospitalizations. There were no predefined subgroups for whom there was seen a significant increase in risk. 
 
\end{document}
